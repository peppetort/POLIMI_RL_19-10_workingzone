\documentclass{article}
\usepackage{hyperref}
\begin{document}
\begin{titlepage}
   \vspace*{\stretch{1.0}}
   \begin{center}
      \Large\textbf{Progetto reti logiche}\\
      \large\textit{Rigutti Luca [codice persona: 10558383]}
      \linebreak
      \large\textit{Tortorelli Giuseppe [codice persona: xxxxxxx]}
   \end{center}
   \vspace*{\stretch{2.0}}
\end{titlepage}

\paragraph{Introduzione}
Il progetto di reti logiche dell'anno accademico 2019-2020 è la realizzazione della codifica Working Zone, metodo pensato per il Bus Indirizzi che usa per trasformare il valore di un indirizzo trasmesso per aumentare l'efficienza.
Il funzionamento del progetto è il seguente:
ci sono 8 Working Zone, che sono gli indirizzi di base, con un intervallo di 4 indirizzi successivi alla base. Quando viene fornito l'indirizzo da codificare, se questo indirizzo si trova in una working zone si codificia nella seguente maniera:
\begin{center}
\begin{tabular}{c c c}
1 (bit per segnalare la codifica) & WZ in binario & OneHot
\end{tabular}
\end{center}
Continuazione della spiegazione
\paragraph{Design}
Nella progettazione del sistema, visto che per accedere alla ram si può farlo solo un ciclo di clock alla volta, si è deciso di memorizzare le working zone all'interno del circuito per poter eseguire la codifica su un ciclo di clock per ogni indirizzo da codificare.
\paragraph{Risultato della simulazione del cicuito}
Qui mettiamo uno screenshot del simulatore
\paragraph{Conclusione}


\end{document}
